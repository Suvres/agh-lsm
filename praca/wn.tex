\chapter{Wnioski}
W trakcie tego projektu zostały wykorzystane biblioteka WebSpeechApi do języka Java Script.\ Jest to bardzo prosta w użyciu biblioteka dająca duże możliwości i prosta w obsłudze.\ Dużym plusem tej biblioteki jest możliwość skorzystania z możliwości przeglądarek do obsługi rozpoznawania i syntezy mowy.\ Pozwoliło to na unowocześnienie aplikacji prostej biblioteki przez dodanie do niej obsługi interfejsu głosowego.\ Dzięki czemu osoby niepełnosprawne z problemami ruchowymi mogą obsługiwać aplikację poprzez wykorzystanie głosu.\ Kolejnym atutem jest możliwość syntezy głosu, dzięki czemu jest możliwość zwracania informacji zwrotnej tak jak zostało to pokazane na formularzu dodawania książek.\ Biblioteka bardzo dobrze sobie radzi z tekstem ciągłym i znakami specjalnymi np. przy dyktowaniu adresu email.\ Jednakże zdania należało wypowiadać bardzo wyraźnie ponieważ samo rozpoznawanie mowy nie zawsze ''rozumiało'' celu wypowiedzi albo słów, wypowiadanych jeśli były robione w sposób niezrozumiały.\ Narzędzia dostarczane przez tą bibliotekę, wskazują jak duży potencjał, mają aplikacje przeglądarkowe, które mogą być uruchamiane bezpośrednio w przeglądarce bądź jako PWA \trans{ang. Progressive Web App} czyli aplikacja ''instalowana'' lokalnie która może w pewnym stopniu działać bez dostępu do internetu.
